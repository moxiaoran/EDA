\documentclass[]{article}
\usepackage{lmodern}
\usepackage{amssymb,amsmath}
\usepackage{ifxetex,ifluatex}
\usepackage{fixltx2e} % provides \textsubscript
\ifnum 0\ifxetex 1\fi\ifluatex 1\fi=0 % if pdftex
  \usepackage[T1]{fontenc}
  \usepackage[utf8]{inputenc}
\else % if luatex or xelatex
  \ifxetex
    \usepackage{mathspec}
  \else
    \usepackage{fontspec}
  \fi
  \defaultfontfeatures{Ligatures=TeX,Scale=MatchLowercase}
\fi
% use upquote if available, for straight quotes in verbatim environments
\IfFileExists{upquote.sty}{\usepackage{upquote}}{}
% use microtype if available
\IfFileExists{microtype.sty}{%
\usepackage{microtype}
\UseMicrotypeSet[protrusion]{basicmath} % disable protrusion for tt fonts
}{}
\usepackage[margin=1in]{geometry}
\usepackage{hyperref}
\hypersetup{unicode=true,
            pdftitle={breats cancer analysis},
            pdfauthor={Yifei Liu},
            pdfborder={0 0 0},
            breaklinks=true}
\urlstyle{same}  % don't use monospace font for urls
\usepackage{color}
\usepackage{fancyvrb}
\newcommand{\VerbBar}{|}
\newcommand{\VERB}{\Verb[commandchars=\\\{\}]}
\DefineVerbatimEnvironment{Highlighting}{Verbatim}{commandchars=\\\{\}}
% Add ',fontsize=\small' for more characters per line
\usepackage{framed}
\definecolor{shadecolor}{RGB}{248,248,248}
\newenvironment{Shaded}{\begin{snugshade}}{\end{snugshade}}
\newcommand{\AlertTok}[1]{\textcolor[rgb]{0.94,0.16,0.16}{#1}}
\newcommand{\AnnotationTok}[1]{\textcolor[rgb]{0.56,0.35,0.01}{\textbf{\textit{#1}}}}
\newcommand{\AttributeTok}[1]{\textcolor[rgb]{0.77,0.63,0.00}{#1}}
\newcommand{\BaseNTok}[1]{\textcolor[rgb]{0.00,0.00,0.81}{#1}}
\newcommand{\BuiltInTok}[1]{#1}
\newcommand{\CharTok}[1]{\textcolor[rgb]{0.31,0.60,0.02}{#1}}
\newcommand{\CommentTok}[1]{\textcolor[rgb]{0.56,0.35,0.01}{\textit{#1}}}
\newcommand{\CommentVarTok}[1]{\textcolor[rgb]{0.56,0.35,0.01}{\textbf{\textit{#1}}}}
\newcommand{\ConstantTok}[1]{\textcolor[rgb]{0.00,0.00,0.00}{#1}}
\newcommand{\ControlFlowTok}[1]{\textcolor[rgb]{0.13,0.29,0.53}{\textbf{#1}}}
\newcommand{\DataTypeTok}[1]{\textcolor[rgb]{0.13,0.29,0.53}{#1}}
\newcommand{\DecValTok}[1]{\textcolor[rgb]{0.00,0.00,0.81}{#1}}
\newcommand{\DocumentationTok}[1]{\textcolor[rgb]{0.56,0.35,0.01}{\textbf{\textit{#1}}}}
\newcommand{\ErrorTok}[1]{\textcolor[rgb]{0.64,0.00,0.00}{\textbf{#1}}}
\newcommand{\ExtensionTok}[1]{#1}
\newcommand{\FloatTok}[1]{\textcolor[rgb]{0.00,0.00,0.81}{#1}}
\newcommand{\FunctionTok}[1]{\textcolor[rgb]{0.00,0.00,0.00}{#1}}
\newcommand{\ImportTok}[1]{#1}
\newcommand{\InformationTok}[1]{\textcolor[rgb]{0.56,0.35,0.01}{\textbf{\textit{#1}}}}
\newcommand{\KeywordTok}[1]{\textcolor[rgb]{0.13,0.29,0.53}{\textbf{#1}}}
\newcommand{\NormalTok}[1]{#1}
\newcommand{\OperatorTok}[1]{\textcolor[rgb]{0.81,0.36,0.00}{\textbf{#1}}}
\newcommand{\OtherTok}[1]{\textcolor[rgb]{0.56,0.35,0.01}{#1}}
\newcommand{\PreprocessorTok}[1]{\textcolor[rgb]{0.56,0.35,0.01}{\textit{#1}}}
\newcommand{\RegionMarkerTok}[1]{#1}
\newcommand{\SpecialCharTok}[1]{\textcolor[rgb]{0.00,0.00,0.00}{#1}}
\newcommand{\SpecialStringTok}[1]{\textcolor[rgb]{0.31,0.60,0.02}{#1}}
\newcommand{\StringTok}[1]{\textcolor[rgb]{0.31,0.60,0.02}{#1}}
\newcommand{\VariableTok}[1]{\textcolor[rgb]{0.00,0.00,0.00}{#1}}
\newcommand{\VerbatimStringTok}[1]{\textcolor[rgb]{0.31,0.60,0.02}{#1}}
\newcommand{\WarningTok}[1]{\textcolor[rgb]{0.56,0.35,0.01}{\textbf{\textit{#1}}}}
\usepackage{longtable,booktabs}
\usepackage{graphicx,grffile}
\makeatletter
\def\maxwidth{\ifdim\Gin@nat@width>\linewidth\linewidth\else\Gin@nat@width\fi}
\def\maxheight{\ifdim\Gin@nat@height>\textheight\textheight\else\Gin@nat@height\fi}
\makeatother
% Scale images if necessary, so that they will not overflow the page
% margins by default, and it is still possible to overwrite the defaults
% using explicit options in \includegraphics[width, height, ...]{}
\setkeys{Gin}{width=\maxwidth,height=\maxheight,keepaspectratio}
\IfFileExists{parskip.sty}{%
\usepackage{parskip}
}{% else
\setlength{\parindent}{0pt}
\setlength{\parskip}{6pt plus 2pt minus 1pt}
}
\setlength{\emergencystretch}{3em}  % prevent overfull lines
\providecommand{\tightlist}{%
  \setlength{\itemsep}{0pt}\setlength{\parskip}{0pt}}
\setcounter{secnumdepth}{0}
% Redefines (sub)paragraphs to behave more like sections
\ifx\paragraph\undefined\else
\let\oldparagraph\paragraph
\renewcommand{\paragraph}[1]{\oldparagraph{#1}\mbox{}}
\fi
\ifx\subparagraph\undefined\else
\let\oldsubparagraph\subparagraph
\renewcommand{\subparagraph}[1]{\oldsubparagraph{#1}\mbox{}}
\fi

%%% Use protect on footnotes to avoid problems with footnotes in titles
\let\rmarkdownfootnote\footnote%
\def\footnote{\protect\rmarkdownfootnote}

%%% Change title format to be more compact
\usepackage{titling}

% Create subtitle command for use in maketitle
\providecommand{\subtitle}[1]{
  \posttitle{
    \begin{center}\large#1\end{center}
    }
}

\setlength{\droptitle}{-2em}

  \title{breats cancer analysis}
    \pretitle{\vspace{\droptitle}\centering\huge}
  \posttitle{\par}
    \author{Yifei Liu}
    \preauthor{\centering\large\emph}
  \postauthor{\par}
      \predate{\centering\large\emph}
  \postdate{\par}
    \date{4/20/2019}


\begin{document}
\maketitle

\begin{Shaded}
\begin{Highlighting}[]
\KeywordTok{library}\NormalTok{(tidyverse)}
\KeywordTok{library}\NormalTok{(broom)}
\KeywordTok{library}\NormalTok{(scales)}
\KeywordTok{library}\NormalTok{(ggcorrplot)}
\KeywordTok{library}\NormalTok{(skimr)}
\KeywordTok{library}\NormalTok{(tidymodels)}
\KeywordTok{library}\NormalTok{(parsnip)}
\KeywordTok{library}\NormalTok{(yardstick)}
\KeywordTok{library}\NormalTok{(modelr)}
\KeywordTok{library}\NormalTok{(Metrics)}
\KeywordTok{library}\NormalTok{(ranger)}
\KeywordTok{library}\NormalTok{(rsample)}
\KeywordTok{library}\NormalTok{(yardstick)}
\KeywordTok{library}\NormalTok{(dplyr)}
\KeywordTok{library}\NormalTok{(ranger)}
\KeywordTok{theme_set}\NormalTok{(}\KeywordTok{theme_minimal}\NormalTok{())}
\end{Highlighting}
\end{Shaded}

load and clean the dataset.

\hypertarget{eda}{%
\subsection{EDA}\label{eda}}

try to visualze the dataset we have, our objective,

\begin{enumerate}
\def\labelenumi{\arabic{enumi}.}
\tightlist
\item
  see the corr between x variables
\item
  see y variable compaision
\item
  how many missing value do we have
\item
  what type of prediction do we need to apply. e.g.~lm, glm, or KNN.
\end{enumerate}

\begin{Shaded}
\begin{Highlighting}[]
\NormalTok{breats_cancer <-}\StringTok{ }\KeywordTok{read_csv}\NormalTok{(}\StringTok{"https://raw.githubusercontent.com/ECE-GitHub/breast-cancer-detection/master/data/breast-cancer-wisconsin.txt"}\NormalTok{, }\DataTypeTok{na =} \KeywordTok{c}\NormalTok{(}\OtherTok{NA}\NormalTok{, }\StringTok{"?"}\NormalTok{, }\StringTok{"No idea"}\NormalTok{, }\StringTok{"#"}\NormalTok{))}
\end{Highlighting}
\end{Shaded}

\begin{verbatim}
## Parsed with column specification:
## cols(
##   Index = col_double(),
##   ID = col_double(),
##   `Clump Thickness` = col_double(),
##   `Uniformity of Cell Size` = col_double(),
##   `Uniformity of Cell Shape` = col_double(),
##   `Marginal Adhesion` = col_double(),
##   `Single Epithelial Cell Size` = col_double(),
##   `Bare Nuclei` = col_double(),
##   `Bland Chromatin` = col_double(),
##   `Normal Nucleoli` = col_double(),
##   Mitoses = col_double(),
##   Class = col_double()
## )
\end{verbatim}

\begin{Shaded}
\begin{Highlighting}[]
\NormalTok{breats_cancer <-}\StringTok{ }\NormalTok{breats_cancer }\OperatorTok
\StringTok{  }\KeywordTok{setNames}\NormalTok{(}\KeywordTok{str_replace_all}\NormalTok{(}\KeywordTok{names}\NormalTok{(breats_cancer), }\StringTok{" "}\NormalTok{, }\StringTok{"_"}\NormalTok{)) }\OperatorTok
\StringTok{  }\KeywordTok{filter}\NormalTok{(Class }\OperatorTok\StringTok{ }\KeywordTok{c}\NormalTok{(}\DecValTok{2}\NormalTok{,}\DecValTok{4}\NormalTok{)) }\OperatorTok
\StringTok{  }\KeywordTok{mutate}\NormalTok{(}\DataTypeTok{Class =} \KeywordTok{as.factor}\NormalTok{(Class)) }\OperatorTok
\StringTok{  }\KeywordTok{select}\NormalTok{(}\OperatorTok{-}\NormalTok{Index, }\OperatorTok{-}\StringTok{ }\NormalTok{ID) }\OperatorTok
\StringTok{  }\KeywordTok{mutate}\NormalTok{(}\DataTypeTok{Class =} \KeywordTok{case_when}\NormalTok{(}
\NormalTok{    Class }\OperatorTok{==}\StringTok{ }\DecValTok{2} \OperatorTok{~}\StringTok{ "benign"}\NormalTok{,}
\NormalTok{    T }\OperatorTok{~}\StringTok{ "malignant"}
\NormalTok{  ),}
  \DataTypeTok{Class =} \KeywordTok{as.factor}\NormalTok{(Class))}
\end{Highlighting}
\end{Shaded}

the objective of this analysis is to help doctor to diagnose breats
cancer, to be specific, to classific whether the result is benign or
malignant, classification problem, now we can use several method in
statistical learning, in supervisousn learning, such as logistical
regress, lm un-supervise learning such as KNN and etc.

But first we need to visualize our data first.

let's take a look at the y variable

\begin{Shaded}
\begin{Highlighting}[]
\NormalTok{breats_cancer }\OperatorTok
\StringTok{  }\KeywordTok{count}\NormalTok{(Class) }\OperatorTok
\StringTok{  }\KeywordTok{ggplot}\NormalTok{(}\KeywordTok{aes}\NormalTok{(Class, n, }\DataTypeTok{fill =}\NormalTok{ Class)) }\OperatorTok{+}
\StringTok{  }\KeywordTok{geom_col}\NormalTok{() }\OperatorTok{+}
\StringTok{  }\KeywordTok{labs}\NormalTok{(}\DataTypeTok{title =} \StringTok{"Number of people diagnose with breast cancer"}\NormalTok{,}
       \DataTypeTok{x =} \StringTok{""}\NormalTok{, }\DataTypeTok{y =} \StringTok{""}\NormalTok{, }\DataTypeTok{legend =} \StringTok{"Class"}\NormalTok{)}
\end{Highlighting}
\end{Shaded}

\includegraphics{breat_cancer_analysis_files/figure-latex/unnamed-chunk-3-1.pdf}

we can see in this the number of malignant is significant higher than
benign (is there a label mistake? I thought most case should be more
benign than mailgnant.). In this case, we can use confusion matrices to
present our result.

Next let's take a look at x variables. see quantile, how many NA do we
have for all the dataset.

\begin{Shaded}
\begin{Highlighting}[]
\NormalTok{breats_cancer }\OperatorTok
\StringTok{  }\KeywordTok{skim}\NormalTok{(Clump_Thickness}\OperatorTok{:}\NormalTok{Mitoses, Class) }\OperatorTok
\StringTok{  }\KeywordTok{pander}\NormalTok{()}
\end{Highlighting}
\end{Shaded}

Skim summary statistics\\
n obs: 15620\\
n variables: 10

\begin{longtable}[]{@{}cccccc@{}}
\caption{Table continues below}\tabularnewline
\toprule
\begin{minipage}[b]{0.11\columnwidth}\centering
variable\strut
\end{minipage} & \begin{minipage}[b]{0.10\columnwidth}\centering
missing\strut
\end{minipage} & \begin{minipage}[b]{0.11\columnwidth}\centering
complete\strut
\end{minipage} & \begin{minipage}[b]{0.08\columnwidth}\centering
n\strut
\end{minipage} & \begin{minipage}[b]{0.11\columnwidth}\centering
n\_unique\strut
\end{minipage} & \begin{minipage}[b]{0.31\columnwidth}\centering
top\_counts\strut
\end{minipage}\tabularnewline
\midrule
\endfirsthead
\toprule
\begin{minipage}[b]{0.11\columnwidth}\centering
variable\strut
\end{minipage} & \begin{minipage}[b]{0.10\columnwidth}\centering
missing\strut
\end{minipage} & \begin{minipage}[b]{0.11\columnwidth}\centering
complete\strut
\end{minipage} & \begin{minipage}[b]{0.08\columnwidth}\centering
n\strut
\end{minipage} & \begin{minipage}[b]{0.11\columnwidth}\centering
n\_unique\strut
\end{minipage} & \begin{minipage}[b]{0.31\columnwidth}\centering
top\_counts\strut
\end{minipage}\tabularnewline
\midrule
\endhead
\begin{minipage}[t]{0.11\columnwidth}\centering
Class\strut
\end{minipage} & \begin{minipage}[t]{0.10\columnwidth}\centering
0\strut
\end{minipage} & \begin{minipage}[t]{0.11\columnwidth}\centering
15620\strut
\end{minipage} & \begin{minipage}[t]{0.08\columnwidth}\centering
15620\strut
\end{minipage} & \begin{minipage}[t]{0.11\columnwidth}\centering
2\strut
\end{minipage} & \begin{minipage}[t]{0.31\columnwidth}\centering
mal: 15164, ben: 456, NA: 0\strut
\end{minipage}\tabularnewline
\bottomrule
\end{longtable}

\begin{longtable}[]{@{}c@{}}
\toprule
\begin{minipage}[b]{0.13\columnwidth}\centering
ordered\strut
\end{minipage}\tabularnewline
\midrule
\endhead
\begin{minipage}[t]{0.13\columnwidth}\centering
FALSE\strut
\end{minipage}\tabularnewline
\bottomrule
\end{longtable}

\begin{longtable}[]{@{}ccccccc@{}}
\caption{Table continues below}\tabularnewline
\toprule
\begin{minipage}[b]{0.30\columnwidth}\centering
variable\strut
\end{minipage} & \begin{minipage}[b]{0.10\columnwidth}\centering
missing\strut
\end{minipage} & \begin{minipage}[b]{0.11\columnwidth}\centering
complete\strut
\end{minipage} & \begin{minipage}[b]{0.08\columnwidth}\centering
n\strut
\end{minipage} & \begin{minipage}[b]{0.07\columnwidth}\centering
mean\strut
\end{minipage} & \begin{minipage}[b]{0.07\columnwidth}\centering
sd\strut
\end{minipage} & \begin{minipage}[b]{0.07\columnwidth}\centering
p0\strut
\end{minipage}\tabularnewline
\midrule
\endfirsthead
\toprule
\begin{minipage}[b]{0.30\columnwidth}\centering
variable\strut
\end{minipage} & \begin{minipage}[b]{0.10\columnwidth}\centering
missing\strut
\end{minipage} & \begin{minipage}[b]{0.11\columnwidth}\centering
complete\strut
\end{minipage} & \begin{minipage}[b]{0.08\columnwidth}\centering
n\strut
\end{minipage} & \begin{minipage}[b]{0.07\columnwidth}\centering
mean\strut
\end{minipage} & \begin{minipage}[b]{0.07\columnwidth}\centering
sd\strut
\end{minipage} & \begin{minipage}[b]{0.07\columnwidth}\centering
p0\strut
\end{minipage}\tabularnewline
\midrule
\endhead
\begin{minipage}[t]{0.30\columnwidth}\centering
Bare\_Nuclei\strut
\end{minipage} & \begin{minipage}[t]{0.10\columnwidth}\centering
16\strut
\end{minipage} & \begin{minipage}[t]{0.11\columnwidth}\centering
15604\strut
\end{minipage} & \begin{minipage}[t]{0.08\columnwidth}\centering
15620\strut
\end{minipage} & \begin{minipage}[t]{0.07\columnwidth}\centering
6.47\strut
\end{minipage} & \begin{minipage}[t]{0.07\columnwidth}\centering
3.22\strut
\end{minipage} & \begin{minipage}[t]{0.07\columnwidth}\centering
1\strut
\end{minipage}\tabularnewline
\begin{minipage}[t]{0.30\columnwidth}\centering
Bland\_Chromatin\strut
\end{minipage} & \begin{minipage}[t]{0.10\columnwidth}\centering
0\strut
\end{minipage} & \begin{minipage}[t]{0.11\columnwidth}\centering
15620\strut
\end{minipage} & \begin{minipage}[t]{0.08\columnwidth}\centering
15620\strut
\end{minipage} & \begin{minipage}[t]{0.07\columnwidth}\centering
4.61\strut
\end{minipage} & \begin{minipage}[t]{0.07\columnwidth}\centering
1.9\strut
\end{minipage} & \begin{minipage}[t]{0.07\columnwidth}\centering
1\strut
\end{minipage}\tabularnewline
\begin{minipage}[t]{0.30\columnwidth}\centering
Clump\_Thickness\strut
\end{minipage} & \begin{minipage}[t]{0.10\columnwidth}\centering
0\strut
\end{minipage} & \begin{minipage}[t]{0.11\columnwidth}\centering
15620\strut
\end{minipage} & \begin{minipage}[t]{0.08\columnwidth}\centering
15620\strut
\end{minipage} & \begin{minipage}[t]{0.07\columnwidth}\centering
7.58\strut
\end{minipage} & \begin{minipage}[t]{0.07\columnwidth}\centering
2.17\strut
\end{minipage} & \begin{minipage}[t]{0.07\columnwidth}\centering
1\strut
\end{minipage}\tabularnewline
\begin{minipage}[t]{0.30\columnwidth}\centering
Marginal\_Adhesion\strut
\end{minipage} & \begin{minipage}[t]{0.10\columnwidth}\centering
0\strut
\end{minipage} & \begin{minipage}[t]{0.11\columnwidth}\centering
15620\strut
\end{minipage} & \begin{minipage}[t]{0.08\columnwidth}\centering
15620\strut
\end{minipage} & \begin{minipage}[t]{0.07\columnwidth}\centering
5.03\strut
\end{minipage} & \begin{minipage}[t]{0.07\columnwidth}\centering
2.91\strut
\end{minipage} & \begin{minipage}[t]{0.07\columnwidth}\centering
1\strut
\end{minipage}\tabularnewline
\begin{minipage}[t]{0.30\columnwidth}\centering
Mitoses\strut
\end{minipage} & \begin{minipage}[t]{0.10\columnwidth}\centering
0\strut
\end{minipage} & \begin{minipage}[t]{0.11\columnwidth}\centering
15620\strut
\end{minipage} & \begin{minipage}[t]{0.08\columnwidth}\centering
15620\strut
\end{minipage} & \begin{minipage}[t]{0.07\columnwidth}\centering
1.66\strut
\end{minipage} & \begin{minipage}[t]{0.07\columnwidth}\centering
1.5\strut
\end{minipage} & \begin{minipage}[t]{0.07\columnwidth}\centering
1\strut
\end{minipage}\tabularnewline
\begin{minipage}[t]{0.30\columnwidth}\centering
Normal\_Nucleoli\strut
\end{minipage} & \begin{minipage}[t]{0.10\columnwidth}\centering
0\strut
\end{minipage} & \begin{minipage}[t]{0.11\columnwidth}\centering
15620\strut
\end{minipage} & \begin{minipage}[t]{0.08\columnwidth}\centering
15620\strut
\end{minipage} & \begin{minipage}[t]{0.07\columnwidth}\centering
4.97\strut
\end{minipage} & \begin{minipage}[t]{0.07\columnwidth}\centering
2.82\strut
\end{minipage} & \begin{minipage}[t]{0.07\columnwidth}\centering
1\strut
\end{minipage}\tabularnewline
\begin{minipage}[t]{0.30\columnwidth}\centering
Single\_Epithelial\_Cell\_Size\strut
\end{minipage} & \begin{minipage}[t]{0.10\columnwidth}\centering
0\strut
\end{minipage} & \begin{minipage}[t]{0.11\columnwidth}\centering
15620\strut
\end{minipage} & \begin{minipage}[t]{0.08\columnwidth}\centering
15620\strut
\end{minipage} & \begin{minipage}[t]{0.07\columnwidth}\centering
4.22\strut
\end{minipage} & \begin{minipage}[t]{0.07\columnwidth}\centering
2.1\strut
\end{minipage} & \begin{minipage}[t]{0.07\columnwidth}\centering
1\strut
\end{minipage}\tabularnewline
\begin{minipage}[t]{0.30\columnwidth}\centering
Uniformity\_of\_Cell\_Shape\strut
\end{minipage} & \begin{minipage}[t]{0.10\columnwidth}\centering
0\strut
\end{minipage} & \begin{minipage}[t]{0.11\columnwidth}\centering
15620\strut
\end{minipage} & \begin{minipage}[t]{0.08\columnwidth}\centering
15620\strut
\end{minipage} & \begin{minipage}[t]{0.07\columnwidth}\centering
5.62\strut
\end{minipage} & \begin{minipage}[t]{0.07\columnwidth}\centering
2.13\strut
\end{minipage} & \begin{minipage}[t]{0.07\columnwidth}\centering
1\strut
\end{minipage}\tabularnewline
\begin{minipage}[t]{0.30\columnwidth}\centering
Uniformity\_of\_Cell\_Size\strut
\end{minipage} & \begin{minipage}[t]{0.10\columnwidth}\centering
0\strut
\end{minipage} & \begin{minipage}[t]{0.11\columnwidth}\centering
15620\strut
\end{minipage} & \begin{minipage}[t]{0.08\columnwidth}\centering
15620\strut
\end{minipage} & \begin{minipage}[t]{0.07\columnwidth}\centering
6.89\strut
\end{minipage} & \begin{minipage}[t]{0.07\columnwidth}\centering
2.47\strut
\end{minipage} & \begin{minipage}[t]{0.07\columnwidth}\centering
1\strut
\end{minipage}\tabularnewline
\bottomrule
\end{longtable}

\begin{longtable}[]{@{}ccccc@{}}
\toprule
\begin{minipage}[b]{0.07\columnwidth}\centering
p25\strut
\end{minipage} & \begin{minipage}[b]{0.07\columnwidth}\centering
p50\strut
\end{minipage} & \begin{minipage}[b]{0.07\columnwidth}\centering
p75\strut
\end{minipage} & \begin{minipage}[b]{0.08\columnwidth}\centering
p100\strut
\end{minipage} & \begin{minipage}[b]{0.13\columnwidth}\centering
hist\strut
\end{minipage}\tabularnewline
\midrule
\endhead
\begin{minipage}[t]{0.07\columnwidth}\centering
3\strut
\end{minipage} & \begin{minipage}[t]{0.07\columnwidth}\centering
8\strut
\end{minipage} & \begin{minipage}[t]{0.07\columnwidth}\centering
10\strut
\end{minipage} & \begin{minipage}[t]{0.08\columnwidth}\centering
10\strut
\end{minipage} & \begin{minipage}[t]{0.13\columnwidth}\centering
▃▃▁▃▂▁▅▇\strut
\end{minipage}\tabularnewline
\begin{minipage}[t]{0.07\columnwidth}\centering
3\strut
\end{minipage} & \begin{minipage}[t]{0.07\columnwidth}\centering
4\strut
\end{minipage} & \begin{minipage}[t]{0.07\columnwidth}\centering
7\strut
\end{minipage} & \begin{minipage}[t]{0.08\columnwidth}\centering
10\strut
\end{minipage} & \begin{minipage}[t]{0.13\columnwidth}\centering
▂▇▂▁▃▅▁▁\strut
\end{minipage}\tabularnewline
\begin{minipage}[t]{0.07\columnwidth}\centering
7\strut
\end{minipage} & \begin{minipage}[t]{0.07\columnwidth}\centering
8\strut
\end{minipage} & \begin{minipage}[t]{0.07\columnwidth}\centering
10\strut
\end{minipage} & \begin{minipage}[t]{0.08\columnwidth}\centering
10\strut
\end{minipage} & \begin{minipage}[t]{0.13\columnwidth}\centering
▁▂▁▂▂▆▅▇\strut
\end{minipage}\tabularnewline
\begin{minipage}[t]{0.07\columnwidth}\centering
3\strut
\end{minipage} & \begin{minipage}[t]{0.07\columnwidth}\centering
4\strut
\end{minipage} & \begin{minipage}[t]{0.07\columnwidth}\centering
7\strut
\end{minipage} & \begin{minipage}[t]{0.08\columnwidth}\centering
10\strut
\end{minipage} & \begin{minipage}[t]{0.13\columnwidth}\centering
▅▇▅▂▃▂▁▆\strut
\end{minipage}\tabularnewline
\begin{minipage}[t]{0.07\columnwidth}\centering
1\strut
\end{minipage} & \begin{minipage}[t]{0.07\columnwidth}\centering
1\strut
\end{minipage} & \begin{minipage}[t]{0.07\columnwidth}\centering
2\strut
\end{minipage} & \begin{minipage}[t]{0.08\columnwidth}\centering
10\strut
\end{minipage} & \begin{minipage}[t]{0.13\columnwidth}\centering
▇▁▁▁▁▁▁▁\strut
\end{minipage}\tabularnewline
\begin{minipage}[t]{0.07\columnwidth}\centering
3\strut
\end{minipage} & \begin{minipage}[t]{0.07\columnwidth}\centering
5\strut
\end{minipage} & \begin{minipage}[t]{0.07\columnwidth}\centering
8\strut
\end{minipage} & \begin{minipage}[t]{0.08\columnwidth}\centering
10\strut
\end{minipage} & \begin{minipage}[t]{0.13\columnwidth}\centering
▇▅▅▅▂▂▅▅\strut
\end{minipage}\tabularnewline
\begin{minipage}[t]{0.07\columnwidth}\centering
3\strut
\end{minipage} & \begin{minipage}[t]{0.07\columnwidth}\centering
4\strut
\end{minipage} & \begin{minipage}[t]{0.07\columnwidth}\centering
4\strut
\end{minipage} & \begin{minipage}[t]{0.08\columnwidth}\centering
10\strut
\end{minipage} & \begin{minipage}[t]{0.13\columnwidth}\centering
▂▇▆▁▂▁▂▂\strut
\end{minipage}\tabularnewline
\begin{minipage}[t]{0.07\columnwidth}\centering
4\strut
\end{minipage} & \begin{minipage}[t]{0.07\columnwidth}\centering
5\strut
\end{minipage} & \begin{minipage}[t]{0.07\columnwidth}\centering
7\strut
\end{minipage} & \begin{minipage}[t]{0.08\columnwidth}\centering
10\strut
\end{minipage} & \begin{minipage}[t]{0.13\columnwidth}\centering
▁▂▇▆▆▅▁▅\strut
\end{minipage}\tabularnewline
\begin{minipage}[t]{0.07\columnwidth}\centering
5\strut
\end{minipage} & \begin{minipage}[t]{0.07\columnwidth}\centering
6\strut
\end{minipage} & \begin{minipage}[t]{0.07\columnwidth}\centering
10\strut
\end{minipage} & \begin{minipage}[t]{0.08\columnwidth}\centering
10\strut
\end{minipage} & \begin{minipage}[t]{0.13\columnwidth}\centering
▁▂▂▃▆▁▃▇\strut
\end{minipage}\tabularnewline
\bottomrule
\end{longtable}

\begin{Shaded}
\begin{Highlighting}[]
\NormalTok{breats_cancer[}\OperatorTok{!}\KeywordTok{complete.cases}\NormalTok{(breats_cancer),]}
\end{Highlighting}
\end{Shaded}

\begin{verbatim}
## # A tibble: 16 x 10
##    Clump_Thickness Uniformity_of_C~ Uniformity_of_C~ Marginal_Adhesi~
##              <dbl>            <dbl>            <dbl>            <dbl>
##  1               1                1                1                1
##  2               1                1                1                1
##  3               3                1                1                1
##  4               8                8                8                1
##  5               5                4                3                1
##  6               6                6                6                9
##  7               3                1                4                1
##  8               1                1                2                1
##  9               3                1                3                1
## 10               1                1                3                1
## 11               8                4                5                1
## 12               3                1                1                1
## 13               1                1                1                1
## 14               1                1                1                1
## 15               5                1                1                1
## 16               4                6                5                6
## # ... with 6 more variables: Single_Epithelial_Cell_Size <dbl>,
## #   Bare_Nuclei <dbl>, Bland_Chromatin <dbl>, Normal_Nucleoli <dbl>,
## #   Mitoses <dbl>, Class <fct>
\end{verbatim}

\begin{Shaded}
\begin{Highlighting}[]
\NormalTok{breats_cancer <-}\StringTok{ }\NormalTok{breats_cancer }\OperatorTok
\StringTok{  }\KeywordTok{na.omit}\NormalTok{()}
\end{Highlighting}
\end{Shaded}

We can do futuer study, like ask question about why all the observations
that have incomplete are missing Bare\_Nuclei value.

Since you have abundant of data, and only hand full of missing data, we
can just filter those data out.

let see the relationship between x variable and y variable

\begin{Shaded}
\begin{Highlighting}[]
\NormalTok{breats_cancer }\OperatorTok
\StringTok{  }\KeywordTok{ggplot}\NormalTok{(}\KeywordTok{aes}\NormalTok{(Class, Clump_Thickness)) }\OperatorTok{+}
\StringTok{  }\KeywordTok{geom_boxplot}\NormalTok{(}\KeywordTok{aes}\NormalTok{(}\DataTypeTok{fill =}\NormalTok{ Class), }\DataTypeTok{alpha =} \FloatTok{0.25}\NormalTok{) }\OperatorTok{+}\StringTok{ }
\StringTok{  }\KeywordTok{geom_jitter}\NormalTok{(}\DataTypeTok{alpha =} \FloatTok{0.5}\NormalTok{, }\DataTypeTok{height =} \FloatTok{0.2}\NormalTok{, }\DataTypeTok{width =} \FloatTok{0.25}\NormalTok{, }\KeywordTok{aes}\NormalTok{(}\DataTypeTok{color =}\NormalTok{ Class)) }\OperatorTok{+}
\StringTok{  }\KeywordTok{scale_color_manual}\NormalTok{(}\DataTypeTok{values =} \KeywordTok{c}\NormalTok{(}\StringTok{"blue"}\NormalTok{, }\StringTok{"red"}\NormalTok{)) }\OperatorTok{+}
\StringTok{  }\KeywordTok{scale_fill_manual}\NormalTok{(}\DataTypeTok{values =} \KeywordTok{c}\NormalTok{(}\StringTok{"blue"}\NormalTok{, }\StringTok{"red"}\NormalTok{)) }\OperatorTok{+}
\StringTok{  }\KeywordTok{labs}\NormalTok{(}\DataTypeTok{title =} \StringTok{"Clump Thickness difference between Benign and mailgnant"}\NormalTok{,}
       \DataTypeTok{x =} \StringTok{""}\NormalTok{, }\DataTypeTok{y =} \StringTok{""}\NormalTok{, }\DataTypeTok{subtitle =} \StringTok{"Score between 1:10"}\NormalTok{)}
\end{Highlighting}
\end{Shaded}

\includegraphics{breat_cancer_analysis_files/figure-latex/unnamed-chunk-5-1.pdf}

Since we have 9 x variable and one one y variable, we can use density
plot show the relationship between x and y variables.

\begin{Shaded}
\begin{Highlighting}[]
\NormalTok{breats_cancer }\OperatorTok
\StringTok{  }\KeywordTok{gather}\NormalTok{(}\DataTypeTok{key =} \StringTok{"key"}\NormalTok{, }\DataTypeTok{value =} \StringTok{"value"}\NormalTok{, }\OperatorTok{-}\NormalTok{Class) }\OperatorTok
\StringTok{  }\KeywordTok{na.omit}\NormalTok{() }\OperatorTok
\StringTok{  }\KeywordTok{ggplot}\NormalTok{(}\KeywordTok{aes}\NormalTok{(value, }\DataTypeTok{fill =} \KeywordTok{as.factor}\NormalTok{(Class))) }\OperatorTok{+}
\StringTok{  }\KeywordTok{geom_density}\NormalTok{() }\OperatorTok{+}
\StringTok{  }\KeywordTok{facet_wrap}\NormalTok{(}\OperatorTok{~}\StringTok{ }\NormalTok{key) }\OperatorTok{+}
\StringTok{  }\KeywordTok{labs}\NormalTok{(}\DataTypeTok{fill =} \StringTok{"Class"}\NormalTok{)}
\end{Highlighting}
\end{Shaded}

\includegraphics{breat_cancer_analysis_files/figure-latex/unnamed-chunk-6-1.pdf}

\begin{Shaded}
\begin{Highlighting}[]
\NormalTok{breats_cancer }\OperatorTok
\StringTok{  }\KeywordTok{gather}\NormalTok{(}\OperatorTok{-}\NormalTok{Class, }\DataTypeTok{value =} \StringTok{"value"}\NormalTok{, }\DataTypeTok{key =} \StringTok{"key"}\NormalTok{) }\OperatorTok
\StringTok{  }\KeywordTok{ggplot}\NormalTok{(}\KeywordTok{aes}\NormalTok{(value, }\DataTypeTok{fill =}\NormalTok{ Class)) }\OperatorTok{+}
\StringTok{  }\KeywordTok{geom_density}\NormalTok{(}\DataTypeTok{position =} \StringTok{"fill"}\NormalTok{, }\DataTypeTok{alpha =} \FloatTok{0.25}\NormalTok{) }\OperatorTok{+}
\StringTok{  }\KeywordTok{facet_wrap}\NormalTok{(}\OperatorTok{~}\StringTok{ }\NormalTok{key) }\OperatorTok{+}
\StringTok{  }\KeywordTok{scale_fill_manual}\NormalTok{(}\DataTypeTok{values =} \KeywordTok{c}\NormalTok{(}\StringTok{"blue"}\NormalTok{, }\StringTok{"red"}\NormalTok{))}
\end{Highlighting}
\end{Shaded}

\includegraphics{breat_cancer_analysis_files/figure-latex/unnamed-chunk-6-2.pdf}

we can see in this density chart, beside single Epithelial Cell size and
Mitosens, all other chart yield similary result, most of these start to
show class as malignanat as those value increase. We can use correlation
plot to identify how many of these variables have a relative higher
correlationship.

let's visualzie the correlationship between all those x variables.

\begin{Shaded}
\begin{Highlighting}[]
\NormalTok{corr <-}\StringTok{ }\NormalTok{breats_cancer }\OperatorTok
\StringTok{  }\KeywordTok{select}\NormalTok{(}\OperatorTok{-}\NormalTok{Class) }\OperatorTok
\StringTok{  }\KeywordTok{setNames}\NormalTok{(}\KeywordTok{str_replace_all}\NormalTok{(}\KeywordTok{names}\NormalTok{(breats_cancer)[}\OperatorTok{-}\DecValTok{10}\NormalTok{], }\StringTok{"_"}\NormalTok{, }\StringTok{" "}\NormalTok{)) }\OperatorTok
\StringTok{  }\KeywordTok{na.omit}\NormalTok{() }\OperatorTok
\StringTok{  }\KeywordTok{cor}\NormalTok{() }\OperatorTok
\StringTok{  }\KeywordTok{round}\NormalTok{(}\DataTypeTok{digits =} \DecValTok{2}\NormalTok{)}


\NormalTok{corr }\OperatorTok
\StringTok{  }\KeywordTok{ggcorrplot}\NormalTok{(}\DataTypeTok{hc.order =}\NormalTok{ T, }\DataTypeTok{type =} \StringTok{"lower"}\NormalTok{,}
             \DataTypeTok{outline.color =} \StringTok{"white"}\NormalTok{,}
             \DataTypeTok{colors =} \KeywordTok{c}\NormalTok{(}\StringTok{"#6D9EC1"}\NormalTok{, }\StringTok{"white"}\NormalTok{, }\StringTok{"#E46726"}\NormalTok{),}
             \DataTypeTok{lab =}\NormalTok{ T) }
\end{Highlighting}
\end{Shaded}

\includegraphics{breat_cancer_analysis_files/figure-latex/unnamed-chunk-7-1.pdf}

let's visualize the x variable distribution relative to y value, since
all the x variable scale from 1:10

\begin{Shaded}
\begin{Highlighting}[]
\NormalTok{breats_cancer }\OperatorTok
\StringTok{  }\KeywordTok{gather}\NormalTok{(}\OperatorTok{-}\NormalTok{Class, }\DataTypeTok{key =} \StringTok{"key"}\NormalTok{, }\DataTypeTok{value =} \StringTok{"value"}\NormalTok{) }\OperatorTok
\StringTok{  }\KeywordTok{mutate}\NormalTok{(}\DataTypeTok{key =} \KeywordTok{str_replace_all}\NormalTok{(key, }\StringTok{"_"}\NormalTok{, }\StringTok{" "}\NormalTok{)) }\OperatorTok
\StringTok{  }\KeywordTok{ggplot}\NormalTok{(}\KeywordTok{aes}\NormalTok{(}\KeywordTok{fct_reorder}\NormalTok{(key, value, }\DataTypeTok{.fun =}\NormalTok{ median), value, }\DataTypeTok{fill =}\NormalTok{ key)) }\OperatorTok{+}
\StringTok{  }\KeywordTok{geom_boxplot}\NormalTok{(}\DataTypeTok{alpha =} \FloatTok{0.25}\NormalTok{, }\DataTypeTok{show.legend =}\NormalTok{ F) }\OperatorTok{+}
\StringTok{  }\KeywordTok{facet_wrap}\NormalTok{(}\OperatorTok{~}\StringTok{ }\NormalTok{Class) }\OperatorTok{+}
\StringTok{  }\KeywordTok{coord_flip}\NormalTok{() }\OperatorTok{+}
\StringTok{  }\KeywordTok{labs}\NormalTok{(}\DataTypeTok{x =} \StringTok{""}\NormalTok{, }\DataTypeTok{y =} \StringTok{""}\NormalTok{,}
       \DataTypeTok{title =} \StringTok{"Value of feature that seperate benign cells and malignant cells"}\NormalTok{,}
       \DataTypeTok{subtitle =} \StringTok{"The ratio of Malignant:Benign = 33:1"}\NormalTok{)}
\end{Highlighting}
\end{Shaded}

\includegraphics{breat_cancer_analysis_files/figure-latex/unnamed-chunk-8-1.pdf}

we can use bootstrap to gague the true median or mean of each variables.

\begin{Shaded}
\begin{Highlighting}[]
\NormalTok{geom_mean <-}\StringTok{ }\ControlFlowTok{function}\NormalTok{(x) \{}
  \KeywordTok{exp}\NormalTok{(}\KeywordTok{mean}\NormalTok{(}\KeywordTok{log}\NormalTok{(x }\OperatorTok{+}\StringTok{ }\DecValTok{1}\NormalTok{)) }\OperatorTok{-}\StringTok{ }\DecValTok{1}\NormalTok{)}
\NormalTok{\}}

\NormalTok{breats_cancer }\OperatorTok
\StringTok{  }\KeywordTok{gather}\NormalTok{(key, value, }\OperatorTok{-}\NormalTok{Class) }\OperatorTok
\StringTok{  }\KeywordTok{bootstraps}\NormalTok{(}\DataTypeTok{times =} \DecValTok{25}\NormalTok{) }\OperatorTok
\StringTok{  }\KeywordTok{unnest}\NormalTok{(}\KeywordTok{map}\NormalTok{(splits, as.data.frame)) }\OperatorTok
\StringTok{  }\KeywordTok{group_by}\NormalTok{(id, key, Class) }\OperatorTok
\StringTok{  }\KeywordTok{summarize}\NormalTok{(}\DataTypeTok{avg_mean =} \KeywordTok{geom_mean}\NormalTok{(value)) }\OperatorTok
\StringTok{  }\KeywordTok{ungroup}\NormalTok{() }\OperatorTok
\StringTok{  }\KeywordTok{mutate}\NormalTok{(}\DataTypeTok{key =} \KeywordTok{fct_reorder}\NormalTok{(key, avg_mean)) }\OperatorTok
\StringTok{  }\KeywordTok{ggplot}\NormalTok{(}\KeywordTok{aes}\NormalTok{(key, avg_mean, }\DataTypeTok{group =}\NormalTok{ key)) }\OperatorTok{+}
\StringTok{  }\KeywordTok{geom_boxplot}\NormalTok{() }\OperatorTok{+}
\StringTok{  }\KeywordTok{coord_flip}\NormalTok{() }\OperatorTok{+}
\StringTok{  }\KeywordTok{facet_wrap}\NormalTok{(}\OperatorTok{~}\StringTok{ }\NormalTok{Class, }\DataTypeTok{scales =} \StringTok{"free_x"}\NormalTok{) }\OperatorTok{+}
\StringTok{  }\KeywordTok{labs}\NormalTok{(}\DataTypeTok{x =} \StringTok{"Mean Score"}\NormalTok{,}
       \DataTypeTok{y =} \StringTok{""}\NormalTok{,}
       \DataTypeTok{title =} \StringTok{""}\NormalTok{)}
\end{Highlighting}
\end{Shaded}

\includegraphics{breat_cancer_analysis_files/figure-latex/unnamed-chunk-9-1.pdf}

Conclusion

\begin{enumerate}
\def\labelenumi{\arabic{enumi}.}
\item
  we have 9 x variables, couple of those varibale have relative high
  correlationship such as, uniformity of cell size, unifority of cell
  shape and normal nucleoli and single epithelial cell size and clump
  thickness and marginal adhesion.
\item
  there are some missing value, but only around 16. So we can just
  filter the data out, you can ask adviors how to deal with it. In my
  case, I just filter out the observations that contain missing values.
\end{enumerate}

\hypertarget{modeling}{%
\subsection{Modeling}\label{modeling}}

Most important question need to answer when building the model

\begin{enumerate}
\def\labelenumi{\arabic{enumi}.}
\tightlist
\item
  How well my model well perform on the new data
\item
  Did I select the best performing model
\end{enumerate}

\hypertarget{traing-test-split}{%
\subsubsection{Traing-test split}\label{traing-test-split}}

80/20 split ratio

\begin{Shaded}
\begin{Highlighting}[]
\KeywordTok{set.seed}\NormalTok{(}\DecValTok{2019}\NormalTok{)}

\NormalTok{cancer_split <-}\StringTok{ }\KeywordTok{initial_split}\NormalTok{(breats_cancer, }\DataTypeTok{prop =} \FloatTok{0.8}\NormalTok{)}

\NormalTok{cancer_train <-}\StringTok{ }\KeywordTok{training}\NormalTok{(cancer_split)}
\NormalTok{cancer_test <-}\StringTok{ }\KeywordTok{testing}\NormalTok{(cancer_split)}
\end{Highlighting}
\end{Shaded}

using training data exclusively to train my model

use cross validation, benefit,

\begin{enumerate}
\def\labelenumi{\arabic{enumi}.}
\tightlist
\item
  this well enable user to use all the training dataset to evaluate
  overall performance of the model
\item
  able to calcuate the multiple measure of the performance. This count
  the nature variation of the performance of the model.
\end{enumerate}

\begin{Shaded}
\begin{Highlighting}[]
\NormalTok{cv_split <-}\StringTok{ }\NormalTok{cancer_train }\OperatorTok
\StringTok{  }\KeywordTok{vfold_cv}\NormalTok{(}\DataTypeTok{v =} \DecValTok{20}\NormalTok{)}
  
\NormalTok{cv_split}\OperatorTok{$}\NormalTok{splits[[}\DecValTok{1}\NormalTok{]]}
\end{Highlighting}
\end{Shaded}

\begin{verbatim}
## <11859/625/12484>
\end{verbatim}

\begin{Shaded}
\begin{Highlighting}[]
\NormalTok{cv_data <-}\StringTok{ }\NormalTok{cv_split }\OperatorTok
\StringTok{  }\KeywordTok{mutate}\NormalTok{(}\DataTypeTok{train =} \KeywordTok{map}\NormalTok{(splits, }\OperatorTok{~}\KeywordTok{training}\NormalTok{(.x)),}
         \DataTypeTok{validate =} \KeywordTok{map}\NormalTok{(splits, }\OperatorTok{~}\KeywordTok{testing}\NormalTok{(.x)))}
\end{Highlighting}
\end{Shaded}

test first model, logisticl regression

\begin{Shaded}
\begin{Highlighting}[]
\NormalTok{cv_models_lr <-}\StringTok{ }\NormalTok{cv_data }\OperatorTok
\StringTok{  }\KeywordTok{mutate}\NormalTok{(}\DataTypeTok{model =} \KeywordTok{map}\NormalTok{(train, }\OperatorTok{~}\KeywordTok{glm}\NormalTok{(Class }\OperatorTok{~}\NormalTok{., }\DataTypeTok{data =}\NormalTok{ .x, }\DataTypeTok{family =} \StringTok{"binomial"}\NormalTok{)))}
\end{Highlighting}
\end{Shaded}

\begin{verbatim}
## Warning: glm.fit: fitted probabilities numerically 0 or 1 occurred
\end{verbatim}

\begin{Shaded}
\begin{Highlighting}[]
\NormalTok{cv_prep_lr <-}\StringTok{ }\NormalTok{cv_models_lr }\OperatorTok
\StringTok{  }\KeywordTok{mutate}\NormalTok{(}\DataTypeTok{validate_actual =} \KeywordTok{map}\NormalTok{(validate, }\OperatorTok{~}\NormalTok{.x}\OperatorTok{$}\NormalTok{Class }\OperatorTok{==}\StringTok{ "malignant"}\NormalTok{),}
         \DataTypeTok{validate_predicted =} \KeywordTok{map2}\NormalTok{(model, validate, }\OperatorTok{~}\KeywordTok{predict}\NormalTok{(.x, .y ,}\DataTypeTok{type =} \StringTok{"response"}\NormalTok{) }\OperatorTok{>}\StringTok{ }\FloatTok{0.5}\NormalTok{))}

\NormalTok{cv_eval_lr <-}\StringTok{ }\NormalTok{cv_prep_lr }\OperatorTok
\StringTok{  }\KeywordTok{mutate}\NormalTok{(}\DataTypeTok{validate_recall =} \KeywordTok{map2_dbl}\NormalTok{(validate_actual, validate_predicted, }\OperatorTok{~}\StringTok{ }\KeywordTok{recall}\NormalTok{(}\DataTypeTok{actual =}\NormalTok{ .x, }\DataTypeTok{predicted =}\NormalTok{ .y)),}
         \DataTypeTok{validate_accuracy =} \KeywordTok{map2_dbl}\NormalTok{(validate_actual, validate_predicted, }\OperatorTok{~}\StringTok{ }\KeywordTok{accuracy}\NormalTok{(}\DataTypeTok{actual =}\NormalTok{ .x, }\DataTypeTok{predicted =}\NormalTok{ .y)),}
         \DataTypeTok{validate_precision =} \KeywordTok{map2_dbl}\NormalTok{(validate_actual, validate_predicted, }\OperatorTok{~}\StringTok{ }\KeywordTok{precision}\NormalTok{(}\DataTypeTok{actual =}\NormalTok{ .x, }\DataTypeTok{predicted =}\NormalTok{ .y)))}

\NormalTok{cv_eval_lr }\OperatorTok
\StringTok{  }\KeywordTok{select}\NormalTok{(validate_recall, validate_accuracy, validate_precision) }\OperatorTok
\StringTok{  }\KeywordTok{setNames}\NormalTok{(}\KeywordTok{c}\NormalTok{(}\StringTok{"recall"}\NormalTok{, }\StringTok{"accuracy"}\NormalTok{, }\StringTok{"precision"}\NormalTok{)) }\OperatorTok
\StringTok{  }\KeywordTok{gather}\NormalTok{() }\OperatorTok
\StringTok{  }\KeywordTok{ggplot}\NormalTok{(}\KeywordTok{aes}\NormalTok{(key, value, }\DataTypeTok{fill =}\NormalTok{ key)) }\OperatorTok{+}
\StringTok{  }\KeywordTok{geom_boxplot}\NormalTok{(}\DataTypeTok{show.legend =}\NormalTok{ F) }\OperatorTok{+}
\StringTok{  }\KeywordTok{scale_y_continuous}\NormalTok{(}\DataTypeTok{labels =}\NormalTok{ scales}\OperatorTok{::}\KeywordTok{percent_format}\NormalTok{()) }\OperatorTok{+}
\StringTok{  }\KeywordTok{expand_limits}\NormalTok{(}\DataTypeTok{y =} \FloatTok{0.995}\NormalTok{) }\OperatorTok{+}
\StringTok{  }\KeywordTok{labs}\NormalTok{(}\DataTypeTok{x =} \StringTok{""}\NormalTok{, }\DataTypeTok{y =} \StringTok{""}\NormalTok{,}
       \DataTypeTok{title =} \StringTok{"Logisticl regression accuracy, precision, recall values"}\NormalTok{)}
\end{Highlighting}
\end{Shaded}

\includegraphics{breat_cancer_analysis_files/figure-latex/unnamed-chunk-12-1.pdf}

we can see the distribution of all these modl are quite stable.

let's use random forst model to predict

\begin{Shaded}
\begin{Highlighting}[]
\NormalTok{cv_tune <-}\StringTok{ }\NormalTok{cv_data }\OperatorTok
\StringTok{  }\KeywordTok{crossing}\NormalTok{(}\DataTypeTok{mtry =} \KeywordTok{c}\NormalTok{(}\DecValTok{3}\NormalTok{,}\DecValTok{5}\NormalTok{,}\DecValTok{7}\NormalTok{,}\DecValTok{9}\NormalTok{))}

\CommentTok{# I only tune one parameter in this case}
\NormalTok{cv_models_rf <-}\StringTok{ }\NormalTok{cv_tune }\OperatorTok
\StringTok{  }\KeywordTok{mutate}\NormalTok{(}\DataTypeTok{model =} \KeywordTok{map2}\NormalTok{(train,mtry, }\OperatorTok{~}\KeywordTok{ranger}\NormalTok{(Class }\OperatorTok{~}\NormalTok{., }\DataTypeTok{data =}\NormalTok{ .x, }\DataTypeTok{mtry =}\NormalTok{ .y, }\DataTypeTok{num.trees =} \DecValTok{100}\NormalTok{,}
                                           \DataTypeTok{seed =} \DecValTok{2019}\NormalTok{)))}

\NormalTok{cv_prep_rf <-}\StringTok{ }\NormalTok{cv_models_rf }\OperatorTok
\StringTok{  }\KeywordTok{mutate}\NormalTok{(}\DataTypeTok{validate_actual =} \KeywordTok{map}\NormalTok{(validate, }\OperatorTok{~}\NormalTok{.x}\OperatorTok{$}\NormalTok{Class }\OperatorTok{==}\StringTok{ "malignant"}\NormalTok{),}
         \DataTypeTok{validate_predicted =} \KeywordTok{map2}\NormalTok{(model, validate, }\OperatorTok{~}\KeywordTok{predict}\NormalTok{(.x, .y ,}\DataTypeTok{type =} \StringTok{"response"}\NormalTok{)}\OperatorTok{$}\NormalTok{predictions }\OperatorTok{==}\StringTok{ "malignant"}\NormalTok{))}

\NormalTok{cv_eval_rf <-}\StringTok{ }\NormalTok{cv_prep_rf }\OperatorTok
\StringTok{  }\KeywordTok{mutate}\NormalTok{(}\DataTypeTok{validate_recall =} \KeywordTok{map2_dbl}\NormalTok{(validate_actual, validate_predicted, }\OperatorTok{~}\StringTok{ }\KeywordTok{recall}\NormalTok{(}\DataTypeTok{actual =}\NormalTok{ .x, }\DataTypeTok{predicted =}\NormalTok{ .y)),}
         \DataTypeTok{validate_accuracy =} \KeywordTok{map2_dbl}\NormalTok{(validate_actual, validate_predicted, }\OperatorTok{~}\StringTok{ }\KeywordTok{accuracy}\NormalTok{(}\DataTypeTok{actual =}\NormalTok{ .x, }\DataTypeTok{predicted =}\NormalTok{ .y)),}
         \DataTypeTok{validate_precision =} \KeywordTok{map2_dbl}\NormalTok{(validate_actual, validate_predicted, }\OperatorTok{~}\StringTok{ }\KeywordTok{precision}\NormalTok{(}\DataTypeTok{actual =}\NormalTok{ .x, }\DataTypeTok{predicted =}\NormalTok{ .y)))}

\NormalTok{cv_eval_rf }\OperatorTok
\StringTok{  }\KeywordTok{select}\NormalTok{(validate_recall, validate_accuracy, validate_precision, mtry) }\OperatorTok
\StringTok{  }\KeywordTok{setNames}\NormalTok{(}\KeywordTok{c}\NormalTok{(}\StringTok{"recall"}\NormalTok{, }\StringTok{"accuracy"}\NormalTok{, }\StringTok{"precision"}\NormalTok{, }\StringTok{"mtry"}\NormalTok{)) }\OperatorTok
\StringTok{  }\KeywordTok{gather}\NormalTok{(key, value, }\OperatorTok{-}\NormalTok{mtry) }\OperatorTok
\StringTok{  }\KeywordTok{ggplot}\NormalTok{(}\KeywordTok{aes}\NormalTok{(key, value, }\DataTypeTok{fill =}\NormalTok{ key)) }\OperatorTok{+}
\StringTok{  }\KeywordTok{geom_boxplot}\NormalTok{(}\DataTypeTok{show.legend =}\NormalTok{ F) }\OperatorTok{+}
\StringTok{  }\KeywordTok{scale_y_continuous}\NormalTok{(}\DataTypeTok{labels =}\NormalTok{ scales}\OperatorTok{::}\KeywordTok{percent_format}\NormalTok{()) }\OperatorTok{+}
\StringTok{  }\KeywordTok{facet_wrap}\NormalTok{(}\OperatorTok{~}\NormalTok{mtry) }\OperatorTok{+}
\StringTok{  }\KeywordTok{expand_limits}\NormalTok{(}\DataTypeTok{y =} \FloatTok{0.995}\NormalTok{) }\OperatorTok{+}
\StringTok{  }\KeywordTok{labs}\NormalTok{(}\DataTypeTok{x =} \StringTok{""}\NormalTok{, }\DataTypeTok{y =} \StringTok{""}\NormalTok{,}
       \DataTypeTok{title =} \StringTok{"Random Forecast model"}\NormalTok{,}
       \DataTypeTok{subtitle =} \StringTok{"differnt mtry"}\NormalTok{)}
\end{Highlighting}
\end{Shaded}

\includegraphics{breat_cancer_analysis_files/figure-latex/unnamed-chunk-13-1.pdf}

base on this two model we can say the accuacy, precision and recall are
quite stable for models. So I use the logisticl regression model as
final model to predict the test dataset. Because it's relartive easy to
interpret.

Let's use

\begin{Shaded}
\begin{Highlighting}[]
\NormalTok{test_actual <-}\StringTok{ }\NormalTok{cancer_test}\OperatorTok{$}\NormalTok{Class }\OperatorTok{==}\StringTok{ "malignant"}

\NormalTok{best_model <-}\StringTok{ }\KeywordTok{glm}\NormalTok{(Class }\OperatorTok{~}\NormalTok{., }\DataTypeTok{data =}\NormalTok{ cancer_test, }\DataTypeTok{family =} \StringTok{"binomial"}\NormalTok{)}
\end{Highlighting}
\end{Shaded}

\begin{verbatim}
## Warning: glm.fit: algorithm did not converge
\end{verbatim}

\begin{verbatim}
## Warning: glm.fit: fitted probabilities numerically 0 or 1 occurred
\end{verbatim}

\begin{Shaded}
\begin{Highlighting}[]
\NormalTok{test_predicted <-}\StringTok{ }\KeywordTok{predict}\NormalTok{(best_model, cancer_test, }\DataTypeTok{type =} \StringTok{"response"}\NormalTok{) }\OperatorTok{>}\StringTok{ }\FloatTok{0.5}

\KeywordTok{table}\NormalTok{(test_actual, test_predicted)}
\end{Highlighting}
\end{Shaded}

\begin{verbatim}
##            test_predicted
## test_actual FALSE TRUE
##       FALSE    83    0
##       TRUE      0 3037
\end{verbatim}

predicted accuracy rate is 100\% for this test dataset.


\end{document}
